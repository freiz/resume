% !TEX program = xelatex

\documentclass[11pt,a4paper,roman,xelatex]{moderncv}

\usepackage{fontspec,xunicode,xltxtra} 

% English font settings
\defaultfontfeatures{Mapping=tex-text, Numbers={OldStyle}}
\setmainfont[Ligatures={Common, Rare}, Alternate=1]{Hoefler Text}
\setsansfont[Scale=0.95]{Optima Regular}
\setmonofont[Scale=0.85]{Source Code Pro}

% Moderncv themes
\moderncvtheme[blue]{classic}

% Adjust the page margins
\widowpenalty=10000
\usepackage[scale=0.81]{geometry}
\AtBeginDocument{\recomputelengths}

\nopagenumbers{}


\firstname{Baojun}
\familyname{Su}
\mobile{+86 18625240866}
\email{freizsu@gmail.com}
\quote{Do everything seriously}

\begin{document}
\maketitle

\section{Educational Background}
\cventry{2008 -- 2011}{Master}{Zhejiang University}{Computer Application Technology}{}{}
\cvline{Research Area: }{\small Large scale online learning, text classification, collaborative filtering algorithm.}
\cventry{2004 -- 2008}{Bachelor}{Jiangsu University of Science and Technology}{Information and Computing Sciences}{}{}

\section{Work Background}
\cventry{2017.2~~ -- now~~}{Sr. Software Engineer}{Microsoft}{Suzhou}{}{Campaign management platform, tech lead}
\cventry{2015.2~~ -- 2017.1~~}{Software Engineer \Romannum{2}}{Microsoft}{Suzhou}{}{Campaign management platform, tech lead}
\cventry{2013.11 -- 2015.1~~}{Software Engineer}{Microsoft}{Suzhou}{}{Distributed key-value datastore}
\cventry{2012.9~~ -- 2013.10}{Software Engineer}{Microsoft}{Beijing}{}{Ad recommandation}
\cventry{2011.4~~ -- 2012.8~~}{Applied Research Engineer}{Netease Youdao}{Beijing}{}{Crawler, web page parser and analyzer, search ranking.}

\section{Technical Ability}
\cvitem{Programming Language}{Java, Python, C, C\#, Matlab, SQL.}
\cvitem{Database Technology}{Microsoft SQL Server, Mongodb, self implemented distributed key-value datastore.}
\cvitem{Machine Learning}{Familiar with common models and algorithms of machine learning and data mining, has in-depth study of online learning, text classification, collaborative filtering, and deep learning.}
\cvitem{Performance Tuning}{Expert in Algorithm performance tuning, service performance turning.}
\cvitem{English}{471 points in CET-6, good English reading and speaking ability, over 5 years experience using English as primary language at work.}

\section{Projects}

  \subsection{AdExtension Framework}
  \cvitem{Introduction}{A dynamic framework to speed up developing new extension types for Bing Ads campaign platform.}
  \cvitem{Timeline}{2015.8 -- 2016.3}
  \cvitem{Duty}{System design, implementation, tech lead.}
  \cvitem{Results}{We implemented dynamic logic using reflection in every components of the service. With static code generation tool, 
  developing new extension type can be as easy as writing a new C\# class. The development time consumption can be reduced from 1/2 year to 1 week.}

  \subsection{Distributed Key-Value Datastore}
  \cvitem{Introduction}{Online serving datastore, serving TB data with low latency and can scale out horizontally, with the ability to plugin user logic as SPROC in C\#.}
  \cvitem{Timeline}{2013.10 -- 2014.9}
  \cvitem{Duty}{System design, implementation, performance tuning.}
  \cvitem{Results}{After migrated the backend of a production service from SQL Server to the new data store, 
  the .95 has been reduced to \emph{1/10} with the same request, and at the same time increased the flexibility and the readability of SPROC.}

  \subsection{Timeliness Web Crawler}
  \cvitem{Introduction}{Crawl news from portal sites, forums, microblogs with low latency.}
  \cvitem{Timeline}{2011.10 -- 2012.4}
  \cvitem{Duty}{Project leader with three team members, architecture design and implementation.}
  \cvitem{Results}{The solution is we first mine the page library to identify possible seed pages, 
  and calculate an estimated refresh interval per seed page then setup a fast flow to crawl the seed page set according to the refresh interval.
  The timeliness coverage rate has been catching up with Baidu after this system going online, compared to nearly no timely results before.}

  \subsection{Efficiency Optimization of DNS Resolve}
  \cvitem{Introduction}{The DNS lookup latency has been becoming the bottleneck of main crawler's efficiency and we try to speedup it.}
  \cvitem{Timeline}{2011.4 -- 2011.6}
  \cvitem{Duty}{Performance optimization.}
  \cvitem{Results}{Upgraded the DNS query logic from multi-thread synchornize call to multi-thread asynchornize call. 
  The DNS lookup job latency has been reduced to less than \emph{1/10} than before.}

  \subsection{GoingMerry (afterhours)}
  \cvitem{Introduction}{A strong programming go (baduk) engine after AlphaGo Zero.}
  \cvitem{Timeline}{2018.1 -- now}
  \cvitem{Results}{It's ongoing developed, the aim is based on AlphaGo Zero, 
  try to find a lightweight solution to avoid huge reinforcement learning computation while have similar or above strength.
  The latest strength is around amateur 6D on KGS using 1600 playouts.}

  \subsection{Terminator (school)}
  \cvitem{Introduction}{A promissing spam filtering library making use of combined machine learnaing algorithms, open sourced at github.}
  \cvitem{Url}{\url{https://github.com/freiz/terminator}}
  \cvitem{Timeline}{2009 -- 2010}
  \cvitem{Results}{Implemented an advanced ensemble algorithm and ensemble 8 machine learning algorithms, can achieve the best results on \emph{All} public datasets, 
  with small memory footprint (default 5~MB) and fast (train/predict hundreds mails in 1 second with 1 thread on normal PC). 
  It won the first place in large-scale spam filtering competition of Eights Symposium of Search Engine and Web Mining and derivated two research papers.} 

\section{Publications}
\cvlistitem{Congfu Xu, Baojun Su, Yunbiao Cheng, Weike Pan, Li Chen, "An Adaptive Fusion Algorithm for Spam Detection", IEEE Intelligent Systems, vol.29, no. 4, pp. 2-8, July-Aug. 2014, doi:10.1109/MIS.2013.54}
\cvlistitem{Baojun Su, Congfu Xu. Not so na\"{i}ve online Bayesian spam filter. In: Proceedings of the 21st conference on Innovative Application of Artificial Intelligence (IAAI 2009), July 14-16, 2009, Pasadena, CA, pages 147-152.}
\cvlistitem{Congfu Xu, Chunliang Hao, Baojun Su. Research on Markov logic networks. Chinese Journal of Software, 2011, 22(8): 1699-1713. (In Chinese with English abstract)}

\end{document}
