\documentclass[11pt,a4paper,roman,xelatex]{moderncv}

\usepackage{fontspec,xunicode,xltxtra} 

% English font settings
\defaultfontfeatures{Mapping=tex-text, Numbers={OldStyle}}
\setmainfont[Ligatures={Common, Rare}, Alternate=1]{Hoefler Text}
\setsansfont[Scale=0.95]{Optima Regular}
\setmonofont[Scale=0.85]{Source Code Pro}

% Moderncv themes
\moderncvtheme[blue]{classic}

% Adjust the page margins
\widowpenalty=10000
\usepackage[scale=0.81]{geometry}
\AtBeginDocument{\recomputelengths}

\nopagenumbers{}


% Personal Data
\firstname{Baojun}
\familyname{Su}
\mobile{+86(010)13439819854}
\email{freizsu@gmail.com}
\quote{Do everything seriously}

% Content
\begin{document}
\maketitle

\section{Educational Background}
\cventry{2008 -- 2011}{Master}{Zhejiang University}{Computer Application Technology}{}{}
\cvline{Research Area: }{\small Large scale online learning, text classification, collaborative filtering algorithm.}
\cventry{2004 -- 2008}{Bachelor}{Jiangsu University of Science and Technology}{Information and Computing Sciences}{}{}

\section{Work Background}
\cventry{2012.9 -- ~~~~~~~~~~}{Software Develop Engineer}{Microsoft STC}{Beijing}{}{Campaign recommendation, bidding and budget suggestion.}
\cventry{2011.4 -- 2012.8}{Applied Research Engineer}{Netease Youdao}{Beijing}{}{Crawler, webpage parser and analyzer, ranking.}

\section{Technical Ability}
\cvitem{Programming Language}{Java, Python, C, Scala, C\#, Matlab, SQL.}
\cvitem{Database Technology}{SQL Server, Mongodb, Kyoto Cabinet, Self implemented DHT.}
\cvitem{Profiling}{Algorithm profiling, service profiling.}
\cvitem{Large-Scale Data Processing}{Experienced with big data processing technology, familiar with several stacks including CoWork/ODFS/OMap in Youdao and Cosmos in Microsoft.}
\cvitem{Data Mining}{Familiar with common models and algorithms of Data Mining, has in-depth study of online learning, text classification and collaborative filtering algorithms.}
\cvitem{English}{471 points in CET-6, good English reading and speaking ability.}
\cvitem{Math}{Solid math foundation.}

\section{Projects}
\subsection{Product Recommendation Service}
\cvitem{Introduction}{Recommend inventories to advertisers in display advertising business.}
\cvitem{Timeline}{2012.9 -- 2013.4}
\cvitem{Duty}{Experiments and algorithms implementation including content-based and collaborative filtering algorithms, and common utility such as evaluation, profiling and instrumentation.}
\cvitem{Results}{The feedbacks from users are mostly positive, the accuracy improved from 66\% to 73\%.}
\subsection{Timeliness Web Crawler}
\cvitem{Introduction}{Crawl news from portal sites, forums, microblogs with low latency.}
\cvitem{Timeline}{2011.10 -- 2012.4}
\cvitem{Duty}{Project leader with three team members.}
\cvitem{Results}{The timeliness coverage rate has been catching up with Baidu.}
\subsection{Efficiency optimization of DNS Resolve}
\cvitem{Introduction}{The DNS lookup rate was becoming the bottleneck of main crawler's efficiency and we try to speedup it.}
\cvitem{Timeline}{2011.4 -- 2011.6}
\cvitem{Duty}{Project leader and the only programmer.}
\cvitem{Results}{The rate of DNS resolving has been sped up by an order of magnitude.}
\subsection{The Application of incremental SVMs in Large-Scale Email Spam Filtering}
\cvitem{Introduction}{National 863 Project}
\cvitem{Timeline}{2008 -- 2009}
\cvitem{Duty}{Project leader, I have built the architecture of our system and implemented most classifiers.}
\cvitem{Results}{The spam filtering system named TGANG is in a leading position internationally both in accuracy and efficiency.}
\subsection{Terminator}
\cvitem{Introduction}{Individual projects, a two-layer ensemble spam filter.}
\cvitem{url}{\url{https://code.google.com/p/terminator/}}
\cvitem{Timeline}{2009 -- 2010}
\cvitem{Results}{Ensemble 8 advanced classifiers, can achieve the best results on \emph{All} public datasets.}

\section{Experience}
\subsection{Scholarship}
\cvitem{2009}{Second Prize Scholarship of Zhejiang University.}
\subsection{Open Source}
\cvitem{2010}{Wrote the NSNB spam filter in Python, which won the first place in large-scale spam filtering competition of Eights Symposium of Search Engine and Web Mining. Now NSNB is open sourced at \url{http://code.google.com/p/nsnb}.}
\subsection{Publications}
\cvlistitem{Baojun Su, Congfu Xu. Not so na\"{i}ve online Bayesian spam filter. In: Proceedings of the 21st conference on Innovative Application of Artificial Intelligence (IAAI 2009), July 14-16, 2009, Pasadena, CA, pages 147-152.}
\cvlistitem{Congfu Xu, Chunliang Hao, Baojun Su. Research on Markov logic networks. Chinese Journal of Software, 2011, 22(8): 1699-1713. (In Chinese with English abstract)}

\section{Self Assessment}
\cvitem{}{I am a quick learner and nice team member, with rich developing experience and good understanding of pattern recognition and machine learning.}

\end{document}
