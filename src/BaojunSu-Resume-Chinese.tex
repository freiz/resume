\documentclass[11pt,a4paper,roman,xelatex]{moderncv}

\usepackage{fontspec,xunicode,xltxtra} 

% English font settings
\defaultfontfeatures{Mapping=tex-text, Numbers={OldStyle}}
\setmainfont[Ligatures={Common, Rare}, Alternate=1]{Hoefler Text}
\setsansfont[Scale=0.95]{Optima Regular}
\setmonofont[Scale=0.85]{Source Code Pro}

% Chinese font settings
\usepackage[slantfont,boldfont,CJKnumber]{xeCJK}
\xeCJKsetup{
    AutoFakeBold=false,
    AutoFakeSlant=false,
}
\setCJKmainfont[BoldFont=STHeiti, ItalicFont={STKaiti}]{STFangsong}
\setCJKsansfont{STHeiti}
\setCJKmonofont{STHeiti}
\setCJKfamilyfont{song}{STSong}
\setCJKfamilyfont{kai}{STKaiti}
\setCJKfamilyfont{hei}{STHeiti}
\setCJKfamilyfont{xinwei}{STXinwei}
\setCJKfamilyfont{fs}{STFangsong}

\newcommand\song{\CJKfamily{song}}
\newcommand\kai{\CJKfamily{kai}}
\newcommand\hei{\CJKfamily{hei}}
\newcommand\xinwei{\CJKfamily{xinwei}}
\newcommand\fs{\CJKfamily{fs}}

\newcommand{\erhao}{\fontsize{22pt}{\baselineskip}\selectfont}
\newcommand{\xiaoerhao}{\fontsize{18pt}{\baselineskip}\selectfont}
\newcommand{\sanhao}{\fontsize{16pt}{\baselineskip}\selectfont}
\newcommand{\xiaosanhao}{\fontsize{15pt}{\baselineskip}\selectfont}
\newcommand{\sihao}{\fontsize{14pt}{\baselineskip}\selectfont}
\newcommand{\xiaosihao}{\fontsize{12pt}{\baselineskip}\selectfont}
\newcommand{\wuhao}{\fontsize{10.5pt}{\baselineskip}\selectfont}
\newcommand{\xiaowuhao}{\fontsize{9pt}{\baselineskip}\selectfont}
\newcommand{\liuhao}{\fontsize{7.5pt}{\baselineskip}\selectfont}
\XeTeXlinebreaklocale "zh"

% moderncv themes
\moderncvtheme[blue]{classic}

% adjust the page margins
\widowpenalty=10000
\usepackage[scale=0.81]{geometry}
\AtBeginDocument{\recomputelengths}

\nopagenumbers{}


\firstname{苏保君}
\familyname{}
\mobile{+86~(512)~18625240866}                    
\email{freizsu@gmail.com}
\quote{认真做好每一件事}                 

\begin{document}
\maketitle

\section{教育背景}
\cventry{2008 -- 2011}{硕士}{浙江大学}{计算机应用技术}{}{}
\cvline{研究方向:}{\small 大规模在线学习、大规模文本分类、推荐系统}
\cventry{2004 -- 2008}{学士}{江苏科技大学}{信息与计算科学}{}{}

\section{工作背景}
\cventry{2012.9 -- ~~~~~~~~~~}{资深软件开发工程师}{微软亚洲搜索技术研发中心}{苏州}{}{在线广告平台、分布式键值数据存储}
\cventry{2011.4 -- 2012.8}{应用研究工程师}{网易有道}{北京}{}{网页搜索的抓取、解析以及购物搜索的排序}

\section{技能}
\cvitem{编程语言}{Java, Python, C, C\#, Matlab, SQL}
\cvitem{数据库}{SQL Server, Mongodb, 分布式存储}
\cvitem{调优}{擅长算法调优、系统调优}
\cvitem{机器学习}{熟悉机器学习、数据挖掘中的常用模型和算法,对在线学习、文本分类和推荐系统方面有深入研究}
\cvitem{英语}{六级471分,具有良好的英文文档阅读及口语表达能力}

\section{项目}
  
  \subsection{分布式键值数据存储}
  \cvitem{介绍}{分布式键值数据存储,在线服务~TB~级数据同时具有较小的延时,性能可以线性水平扩展,可插入~SPROC~执行逻辑}
  \cvitem{时间}{2013.10 -- 2014.9}
  \cvitem{职责}{系统设计、部分实现、调优}
  \cvitem{效果}{对比之前线上~SQL Server~系统,同样的请求~.95~减小了10倍,同时提高了灵活性和~SPROC~的可读性}

  \subsection{时效性网页抓取}
  \cvitem{介绍}{分钟级的延迟抓取新闻、论坛、微博等时效性信息}
  \cvitem{时间}{2011.10 -- 2012.4}
  \cvitem{职责}{系统设计、调优}
  \cvitem{效果}{有道的时效性结果从上线前的几乎无结果到上线后后台覆盖率接近百度的水平}

  \subsection{DNS~查询效率优化}
  \cvitem{介绍}{优化前~DNS~解析吞吐量太小,不能满足大规模抓取需要}
  \cvitem{时间}{2011.4 -- 2011.6}
  \cvitem{职责}{优化~DNS~系统查询效率}
  \cvitem{效果}{通过将底层~DNS~解析从同步~IO~改为异步~IO,查询效率提高了~10~倍}

  \subsection{Terminator}
  \cvitem{介绍}{个人项目,两层集成垃圾邮件过滤器}
  \cvitem{项目地址}{\url{htt://github.com/freiz/terminator}}
  \cvitem{时间}{2009 -- 2010}
  \cvitem{效果}{实现了自创的在线集成算法,集成了8个优秀分类器,在所有公开数据集上都取得最优结果,被某outlook插件作为算法核心}

\section{经历}
\cvitem{2010}{使用~Python~独立开发了~NSNB~邮件过滤算法,并以此获得了~SEWM2010~(第八届全国搜 索引擎和网上信息挖掘学术研讨会)大规模垃圾邮件过滤比赛综合第一名,NSNB~项目目前开源,项目地址是: \url{http://code.google.com/p/nsnb/}}
\cvitem{2010}{所开发的《基于内容的多分类器多层垃圾邮件实时过滤系统软件》获得了计算机软件著 作版权登记证书}

\section{发表论文}
\cvlistitem{Congfu Xu, Baojun Su, Yunbiao Cheng, Weike Pan, Li Chen, "An Adaptive Fusion Algorithm for Spam Detection", IEEE Intelligent Systems, vol.29, no. 4, pp. 2-8, July-Aug. 2014, doi:10.1109/MIS.2013.54}
\cvlistitem{Baojun Su, Congfu Xu. Not so na\"{i}ve online Bayesian spam filter. In: Proceedings of the 21st conference on Innovative Application of Artificial Intelligence (IAAI 2009), July 14-16, 2009, Pasadena, CA, pages 147-152.}
\cvlistitem{Congfu Xu, Chunliang Hao, Baojun Su. Research on Markov logic networks. Chinese Journal of Software, 2011, 22(8): 1699-1713. (In Chinese with English abstract)}

\end{document}
